\documentclass[a4paper]{article}

\usepackage[spanish]{babel}
\usepackage[utf8]{inputenc}
\usepackage{amsmath}
\usepackage{graphicx}
\usepackage{epigraph}
\usepackage[colorinlistoftodos]{todonotes}
\setlength{\parskip}{3mm}
\title{Crisis de los fundamentos y su relación con el inicio de la computación}
\setlength\parindent{0pt}
\usepackage{times}		
\usepackage[T1]{fontenc}

\author{Sebastián Marulanda Quiceno}

\date{27 De Marzo Del 2020}

\begin{document}
\maketitle

\begin{abstract}
Este ensayo busca relatar la historia de la computacion y la crisis de los fundamentos, dando a mostrar como nació la computacion moderna.

\end{abstract}

\epigraph{Si no sabes una verdad en su totalidad, entonces eres preso de una mentira.}{Proverbio de autor anonimo}


\section{Introduction}

Desde hace varios siglos el lenguaje de las matemáticas ha sido uno de los principales pilares para llevar a cabo todos los proyectos, avances e ideas que son planteadas por el ser humano. Sin embargo, es una herramienta que, aunque bien sea de gran utilidad también deja bastantes incógnitas a ser estudiadas. Por este motivo, a mediados del siglo XX surgió la llamada “crisis de los fundamentos”, la cual se basa en que en la decaída de la civilización griega vino un largo período de estancamiento en el desarrollo científico, en particular en las matemáticas. Esto, salvo algunas significativas contribuciones como las hechas por los indios y los árabes. Debido a cambios sociales y a la conquista de las distancias, el pensamiento volvió a ser reflexivo, pero esta vez más ligado a la solución de problemas concretos, al conocimiento de las leyes físicas del universo en base a nuevos modelos matemáticos.


\section{Relato sobre la computación}
\label{sec:examples}

\subsection{Historia de la computación}

Esta crisis fue tema de conversación entre varios autores, que intentaron darle nuevas y mejoradas bases de desarrollo. Unos años atrás, la crisis de los fundamentos había dividido a toda la comunidad científica en varias partes, Una de ellas, eran los formalistas, que decían que todo era posible con las matemáticas si se le dedicaba el tiempo suficiente. Un matemático de tremenda reputación que abanderaba el movimiento, resumió todo el movimiento de los formalistas con una frase: “Debemos saber y sabremos”. Aspiraban a refundar las bases (o axiomas) de las matemáticas para evitar las paradojas planteadas, derivadas, seguramente, de un error o falta de precisión en los planteamientos.

El programa de HIbert que proponía un punto de vista matemático desde un nivel superior para demostrar que los sistemas axiomáticos bien definidos tenían tres propiedades que los convertiría infalibles, la primera propiedad era que estos sistemas eran consistentes ósea, no tenían contradicciones (No se contradecía a si mismo, diciendo que un sistema axiomático era falso y verdadero al mismo tiempo). Otra propiedad era que eran finitarios, de forma que las demostraciones se podían llevar a cabo siguiendo una secuencia de pasos lógicos, de una forma algorítmica y que terminaba en algún momento, y la ultima propiedad era que estos sistemas eran completos, ósea que para cada afirmación del sistema se podría decir que era verdadero o que era falsa.

Después, un matemático austríaco “Gödel” hizo una afirmación diciendo que estaba a punto de demostrar que un sistema no podría ser consistente, finitario, y completo, ósea, decía el que el programa de Hilbert era imposible de concluir. Un año después publico el articulo sobre proposiciones formalmente indecidibles de principio matemática y sistemas relacionados, el matemático demostró, de una forma muy compleja y tremendamente minuciosa su primer teorema de incompletitud, que le quito veracidad al programa de Hilbert, esta demostración marco un punto de inflexión en la historia de las matemáticas.

Alan Turing siguió con el legado de este matemático “Gödel” que tiro a la basura el programa de Hilbert. Alan Turing, aunque ahora se le reconoce mas que todo por su colaboración en la segunda guerra mundial, en la cual descifró los mensajes nazis codificados con la maquina “enigma”, utilizo un razonamiento con parecidos al artículo del matemático “Gödel” para solucionar el llamado Entscheidungsproblem o “problema de decisión”, este dice que, en todos los sistemas, no siempre es posible determinar con un numero finito de pasos si un problema escogido al azar tiene o no tiene solución.

\subsection{Conclucion}

Turing probó que este simple mecanismo o conjunto de ellos, podría resolver toda tarea algorítmica presentada, podría sumar, restar, multiplicas y hacer cualquier tarea basada en una repetición de pasos. Así la aparente barrera que había descubierto Gödel no solo recorto el potencial de las matemáticas, sino que ayudo a imaginar la maquina que mas limites ha hecho saltar a la humanidad

\end{document}